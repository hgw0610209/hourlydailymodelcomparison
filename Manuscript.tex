% Options for packages loaded elsewhere
\PassOptionsToPackage{unicode}{hyperref}
\PassOptionsToPackage{hyphens}{url}
%
\documentclass[
  12,
]{article}
\usepackage{lmodern}
\usepackage{amssymb,amsmath}
\usepackage{ifxetex,ifluatex}
\ifnum 0\ifxetex 1\fi\ifluatex 1\fi=0 % if pdftex
  \usepackage[T1]{fontenc}
  \usepackage[utf8]{inputenc}
  \usepackage{textcomp} % provide euro and other symbols
\else % if luatex or xetex
  \usepackage{unicode-math}
  \defaultfontfeatures{Scale=MatchLowercase}
  \defaultfontfeatures[\rmfamily]{Ligatures=TeX,Scale=1}
\fi
% Use upquote if available, for straight quotes in verbatim environments
\IfFileExists{upquote.sty}{\usepackage{upquote}}{}
\IfFileExists{microtype.sty}{% use microtype if available
  \usepackage[]{microtype}
  \UseMicrotypeSet[protrusion]{basicmath} % disable protrusion for tt fonts
}{}
\makeatletter
\@ifundefined{KOMAClassName}{% if non-KOMA class
  \IfFileExists{parskip.sty}{%
    \usepackage{parskip}
  }{% else
    \setlength{\parindent}{0pt}
    \setlength{\parskip}{6pt plus 2pt minus 1pt}}
}{% if KOMA class
  \KOMAoptions{parskip=half}}
\makeatother
\usepackage{xcolor}
\IfFileExists{xurl.sty}{\usepackage{xurl}}{} % add URL line breaks if available
\IfFileExists{bookmark.sty}{\usepackage{bookmark}}{\usepackage{hyperref}}
\hypersetup{
  pdftitle={Hourly vs.~Daily Pollution Modelling Strategies in Epidemiological Studies: Evaluating the Performance for Estimating Daily Pollution Exposure},
  pdfauthor={Guowen Huang},
  hidelinks,
  pdfcreator={LaTeX via pandoc}}
\urlstyle{same} % disable monospaced font for URLs
\usepackage{graphicx}
\makeatletter
\def\maxwidth{\ifdim\Gin@nat@width>\linewidth\linewidth\else\Gin@nat@width\fi}
\def\maxheight{\ifdim\Gin@nat@height>\textheight\textheight\else\Gin@nat@height\fi}
\makeatother
% Scale images if necessary, so that they will not overflow the page
% margins by default, and it is still possible to overwrite the defaults
% using explicit options in \includegraphics[width, height, ...]{}
\setkeys{Gin}{width=\maxwidth,height=\maxheight,keepaspectratio}
% Set default figure placement to htbp
\makeatletter
\def\fps@figure{htbp}
\makeatother
\setlength{\emergencystretch}{3em} % prevent overfull lines
\providecommand{\tightlist}{%
  \setlength{\itemsep}{0pt}\setlength{\parskip}{0pt}}
\setcounter{secnumdepth}{5}
\usepackage{subcaption}
\makeatletter\@ifundefined{subfloat}{\newcommand{\subfloat}[2][need a sub-caption]{\subcaptionbox{#1}{#2}\, }}{ \renewcommand{\subfloat}[2][need a sub-caption]{ \subcaptionbox{#1}{#2}\, }}\makeatother
\usepackage{xcolor}
\usepackage{booktabs}
\usepackage{setspace}\doublespacing
\usepackage{float}
\usepackage[a4paper,width=15cm, lines=26, left=3cm ]{geometry}
\usepackage[left]{lineno}

\ifluatex
  \usepackage{selnolig}  % disable illegal ligatures
\fi
\usepackage[style=authoryear,backend=biber,maxbibnames = 20,maxcitenames
= 2,doi = false,isbn = false,giveninits = true,uniquelist =
false]{biblatex}
\addbibresource{thebib.bib}

\title{Hourly vs.~Daily Pollution Modelling Strategies in
Epidemiological Studies\(:\) Evaluating the Performance for Estimating
Daily Pollution Exposure}
\author{Guowen Huang}
\date{}

\begin{document}
\maketitle

\hypertarget{abstract}{%
\subsection*{Abstract}\label{abstract}}
\addcontentsline{toc}{subsection}{Abstract}

\linenumbers

Epidemiological studies have consistently demonstrated the association
between daily pollution exposure and disease outcomes. To estimate daily
exposure, hourly pollution data are commonly aggregated, but missing
data pose a significant challenge to this approach. To overcome this
issue, some researchers develop models to impute missing hourly data.
Alternatively, directly modelling pollution exposure on a daily basis is
possible, thereby avoiding the computational burden of hourly pollution
modelling. However, the performance of these two modelling strategies
remains unclear. This study conducts a comparative assessment between
hourly and daily modelling strategies for the purpose of estimating
daily pollution exposure. Utilizing data derived from Guangzhou city,
the analysis encompasses diverse scenarios of data absence. The outcomes
consistently highlight the superior performance of daily pollution
models in terms of mitigated bias and diminished root mean square error
(RMSE) values. These findings can guide researchers in selecting the
most suitable modelling approach for estimating pollution exposure in
epidemiological studies.

\hypertarget{keywords}{%
\subsection*{Keywords}\label{keywords}}
\addcontentsline{toc}{subsection}{Keywords}

Bayesian hierarchical model; modelling comparison; exposure estimation;
MCMC; Stan

\hypertarget{author-information}{%
\subsection*{Author information}\label{author-information}}
\addcontentsline{toc}{subsection}{Author information}

\(^1\) Department of Mathematics, Shantou University, Guangdong 515063,
P.R.China

\(^*\) Corresponding author: E-mail: guowenhuang@stu.edu.cn

\hypertarget{introduction}{%
\section{Introduction}\label{introduction}}

Air pollution is a critical global issue with severe health
consequences. Extensive epidemiological studies have linked daily
pollution exposure to adverse health outcomes, focusing on particulate
matter (PM\(_{2.5}\), PM\(_{10}\)), nitrogen dioxide (NO\(_2\)), and
ozone (O\(_3\)). These pollutants have been extensively studied due to
their prevalence in the atmosphere and associations with respiratory and
cardiovascular diseases, as well as premature mortality among the
elderly. Over the past few decades, the recognition of air pollution as
a worldwide problem has led to a proliferation of epidemiological
research, estimating the health impact of outdoor air pollution.

A report from the World Health Organization (WHO) claimed that 99\% of
the world population was living in places where the WHO air quality
guidelines levels were not met in 2019, and ambient (outdoor) air
pollution in both cities and rural areas was estimated to cause 4.2
million premature deaths worldwide in 2016 \autocite{WHO2021}. According
to European Environment Agency \autocite{EEA2021}, in 2019 air pollution
continued to drive a significant burden of premature death and disease
in the 27 EU Member States: 307,000 premature deaths were attributed to
chronic exposure to PM\(_{2.5}\); 40,400 premature deaths were
attributed to chronic NO\(_2\) exposure; 16,800 premature deaths were
attributed to acute O\(_3\) exposure. The adverse human health effects
associated with exposures to air pollutants have been well documented.
For example, mortality from respiratory and heart diseases was
significantly related to the level of air pollution \autocites[see
e.g.,][]{Hoek2013,Hodgson1970}, while all-cause mortality was found to
be significantly associated with O\(_3\) levels \autocites[see
e.g.,][]{Huang2022,Anderson1996}. \textcite{Lee2020} found that
increased levels of air pollution were consistently associated with
increased risks of respiratory disease (both hospitalization and
mortality). A recent study by \textcite{Zhang2021} found that long-term
exposure to PM\(_{2.5}\) and NO\(_2\) was associated with increased
risks for non-accidental, cardiovascular, and respiratory mortality.

While estimating short-term air pollution effects, daily counts of
disease cases are often regressed against daily air pollution
concentrations on the preceding few days, utilizing either an ecological
(at the population level) time series design \autocites[see
e.g.,][]{Zhou2021,Krall2013,Gouveia2000} or a case-crossover design
\autocites[see
e.g.,][]{Blangiardo2023,Lavigne2022,Liu2019,Di2017,Wing2017}. However,
pollution data often comprise hourly monitoring observations,
occasionally marred by missing values stemming from technical glitches
or equipment malfunctions. In response to this challenge, a common
strategy involves the aggregation of hourly pollution data into daily
values through the computation of means based on the observed hourly
measurements. However, it's essential to recognize that this methodology
hinges on the assumption that the absence of hourly data follows a
random pattern. Yet, this assumption may not universally hold true,
potentially giving rise to bias in the calculated daily exposure values.
In situations where certain monitoring stations, such as rural ones that
traditionally record lower pollution levels, experience missing hourly
data, the act of excluding these gaps during the daily averaging process
could inadvertently skew the daily exposure estimates.

The imperative of achieving accurate exposure estimation has attracted
substantial attention within the realm of existing research. A
noteworthy illustration of this notion can be found in a study conducted
by \textcite{Butland2019}, which underscores the pivotal role played by
measurement precision. This investigation unveiled the propensity for
measurement errors to attenuate projected health impacts, consequently
undermining the reliability of health-related inferences drawn from the
data. Similarly, \textcite{huang2016glaphd} reported analogous findings.
\textcite{Cefalu2014} demonstrated that the bias inherent in
health-effect estimates is influenced by the nuances of the exposure
prediction model. The importance of precisely gauging pollution exposure
in epidemiological studies is accentuated by these findings. By doing
so, a more profound comprehension of the health consequences of air
pollution can be attained. As a result, certain researchers have sought
alternative avenues, developing diverse pollution models to infer
missing hourly data and then aggregating them to get daily exposure.
These methods include the kriging interpolation techniques \autocite[see
e.g.,][]{Cressie1993}, artificial neural network approaches
\autocite[see e.g.,][]{Cordova2021}, and univariate or multivariate
time-series imputation strategies \autocite[see e.g.,][]{Hadeed2020}.
These models have been demonstrated to enhance the accuracy of daily
pollution exposure estimation when compared to simple daily averaging.

When encountering missing data, an alternative approach to estimating
daily pollution exposure is to directly model pollution exposure on a
daily basis. By adopting this strategy, the intricate computational
challenges associated with hourly pollution modelling can be
circumvented. In this daily-centric modelling approach, the task entails
the estimation of pollution concentrations for each individual day,
leveraging not only the available monitoring data but also augmenting it
with additional variables (if possible) such as meteorological
measurements and land-use characteristics. This method holds the
inherent advantage of simplifying the modelling workflow, while the
additional variables could potentially mitigate the uncertainties that
typically arise from imputing absent daily concentration values.

Despite the potential advantages and disadvantages of these two
modelling strategies, the performance of hourly and daily modelling
strategies in estimating pollution exposure remains unclear. In this
paper, we aim to compare the performance of hourly and daily modelling
strategies in estimating daily pollution exposure, using data from
Guangzhou city, China. We hypothesize that daily pollution modelling may
provide comparable results to hourly pollution modelling while saving
computation time. The results of this study will have important
implications for epidemiological studies of estimating air pollution
exposure, enhancing our understanding of the health impacts of air
pollution. Additionally, our findings may inform the development of new
methods for estimating pollution exposure that combine the strengths of
hourly and daily modelling strategies.

\hypertarget{methodology}{%
\section{Methodology}\label{methodology}}

\hypertarget{related-work}{%
\subsection{Related work}\label{related-work}}

The main sources of air pollution data in epidemiological studies
include: (a) Monitoring Stations, which are strategically located in
various urban and rural areas to measure pollutant concentrations in the
ambient air; (b) Satellite Observations, where satellite-based remote
sensing is utilized to monitor air pollution on a larger spatial scale;
(c) Modelling and Simulation, where air pollution models use
mathematical algorithms and computer simulations to estimate pollutant
concentrations based on emission inventories, meteorological data, and
other relevant factors.

For the purpose of this study, our focus is on addressing missing data
specifically from source (a), which refers to the monitoring stations.
To handle this issue, researchers commonly assume missing at random
\autocite[MAR, see e.g.,][]{Little2019} which is frequently encountered
in environmental health sciences studies, and employs two main
techniques \autocite{Hadeed2020}: univariate time-series and
multivariate time-series methods. These approaches have been extensively
utilized to impute missing data in monitoring observations. Univariate
time-series imputation refers to a class of techniques employed in air
pollution studies to address the temporal characteristics of real-time
monitoring data \autocite{Moritz2015}. One commonly used method is the
last observation carried forward (LOCF), which fills gaps of missing
data by using the most recent observed value
\autocite{Engels2003,Plaia2006}. By employing univariate time-series
methods, researchers can estimate missing values by considering the
historical trends and patterns within the specific pollutant's time
series. These techniques utilize the available data points to predict
and fill in the gaps. In contrast, multivariate time-series imputation
involves more sophisticated techniques that take into account multiple
variables and their interactions. By leveraging the relationships
between various pollutants and environmental factors, these methods can
better estimate missing data points based on the available information.
Some commonly employed methods in this category include regression
imputation and predictive mean matching \autocites[PMM, see
e.g.,][]{Rubin1986,Little1988}, row mean method \autocite[RMM, see
e.g.,][]{Engels2003} , multiple imputation chained equations
\autocite[MICE, see e.g.,][]{Rubin1988} and random imputation
\autocite{Moritz2015}.

In a recent study, \textcite{Huang2022} proposed a novel multivariate
time-series model that uses a Bayesian framework to impute hourly
missing pollution data. Note that the Bayesian hierarchical modelling
has gained more focus in the last decades, since the model's versatility
in accommodating intricate data structures aligns well with the
complexities inherent in dataset and it can incorporate prior
information effectively. In their proposed hourly model, the response is
a mixture of Gamma and Half-Cauchy distributions, and the correlations
between pollutants are allowed to vary seasonally. The imputation
performance of this model is exceptional, and its source code is openly
accessible, making it highly suitable for replication by other
researchers. Leveraging this hourly pollution model as a basis for
comparison in our study, we devised a novel daily pollution model. The
aim of this study is to compare the performance of the hourly and daily
pollution models (based on the same dataset) to determine if the daily
pollution model is sufficient, thus avoiding using the computationally
expensive hourly pollution model. We will briefly describe the hourly
pollution model here \autocite[more details can be found
in][]{Huang2022} before presenting our proposed daily pollution model.

\hypertarget{hourly-pollution-model}{%
\subsection{Hourly pollution model}\label{hourly-pollution-model}}

The pollution measurements exhibit a non-negative and right-skewed
distribution, which makes the Gamma distribution a suitable choice for
representing the response variable. One key advantage of the Gamma
distribution over the commonly used log-Normal distribution is that the
sum of Gamma-distributed variables follows a Gamma distribution as well.
This property plays a crucial role in data reduction as discussed in
\textcite{Huang2022}.

Let's denote the measured variable for pollutant \(p\) at monitoring
station \(s\) on hour \(h\) of day \(t\) as \(Y_{sthp}\). This variable
can take two forms: either a clean data point following a Gamma
distribution denoted as \(Y_{sthp}^{[\mbox{clean}]}\), or a contaminated
observation with a scaled half-Cauchy distribution denoted as
\(Y_{sthp}^{[\mbox{cont}]}\). Suspicious outliers in the pollution data
include two types: occasional very high values and isolated zero values
on days when all other stations have non-zero measurements. As shown by
\textcite{He2022} that the observed probability density functions of
pollutant concentrations generically exhibit heavy tails, such as nitric
oxide (NO), NO\(_2\), PM\(_{2.5}\) and PM\(_{10}\). Therefore, the
half-Cauchy distribution is used to accommodate the heavy right tail for
the former outliers and a peak at zero for the latter outliers.

To indicate whether each observation is clean or contaminated, an
indicator variable \(Z_{sthp}\) is introduced. With this notation, the
model can be expressed as follows:

\begin{equation}\label{eqnPollutionhour}
\begin{aligned}
Y_{sthp} \mid Z_{sthp}  = & 
\begin{cases}
Y_{sthp}^{[\mbox{clean}]} &  Z_{sthp} =1 \\ 
Y_{sthp}^{[\mbox{cont}]} & Z_{sthp}=0
\end{cases} \\
Y_{sthp}^{[\mbox{clean}]} \sim & \mbox{Gamma}
\left( \alpha_p, \alpha_p / \lambda_{sthp} \right) \\
Y_{sthp}^{[\mbox{cont}]} \sim & \mbox{Half-Cauchy}(\nu_{p})\\
Z_{sthp} \sim & \mbox{Bern}(\rho_{pt})
\end{aligned}
\end{equation}

The parameter \(\rho_{pt}\), representing the prevalence of
`contaminated' observations, is dynamic and can vary over time to
incorporate advancements in monitoring equipment or significant changes
in the environment that may affect the occurrence of contaminated
observations. These changes can result in an increase or decrease in the
prevalence of contaminated observations. In our modelling approach, we
consider \(\rho_{pt}\) to be piecewise-constant, with a single step
occurring on 1st January 2020. This particular step point is selected to
approximately align with the onset of the COVID-19 pandemic
\autocite{whotimeline}, aiming to minimize potential bias that may arise
due to the unique circumstances during that period.

The clean measurements \(Y_{sthp}^{[\mbox{clean}]}\) follow a Gamma
distribution characterized by shape parameters \(\alpha_p\) and mean
values \(\lambda_{sthp}\). The means are modelled using a log-linear
approach, as described below. \begin{equation}\label{eqnPollutionhour2}
\begin{aligned}
\log{(\lambda_{sthp})}  = & \symbf{X}_{th}^{\top}\symbf{\psi}_{p} +Q_{sp}+ D_{tp} + C_{stp}\\
\symbf{Q}_{s} \sim & \mbox{N}\left(\symbf{0}, \Omega \right)\\
\symbf{D}_{t}  \mid \symbf{D}_{t-1}   \sim & \mbox{N}\left[\mbox{diag}(\symbf{\phi})  \symbf{D}_{t-1} ,\ \symbf{\Sigma}_t \right]\\
\symbf{C}_{st} \sim & \mbox{N}\left[\symbf{0},\ \mbox{diag}(\symbf{\gamma}) \symbf{\Sigma}_t \mbox{diag}(\symbf{\gamma}) \right] \\
\symbf{\Sigma}_t  = &  \mbox{diag}(\symbf{\delta}) \left[
\begin{array}{cccc}
&  &  &  \\
& \symbf{\bar{L}}  &   &  {\symbf{0}}  \\ 
&  &  &  \\
L_{t1} & \dots & L_{t,P-1}  &  L_{tP}
\end{array}\right] \left[
\begin{array}{cccc} 
&  &  &  L_{t1} \\
& \symbf{\bar{L}}^{\top}  &   &  \dots  \\ 
&  &  & L_{t,P-1} \\
& {\symbf{0}}^{\top} &   &  L_{tP}
\end{array}\right] \mbox{diag}(\symbf{\delta}) \\
L_{tp}  = &  b_p + \symbf{f}_t ^{\top}\symbf{\eta}_p
\end{aligned}
\end{equation}

The covariates \(\symbf{X}_{th}\) consist of sinusoidal functions that
capture seasonality at frequencies of 6 and 12 months, indicator
variables for each hour of the day, day-of-week indicators, and a linear
time trend. The fixed effect coefficients \(\symbf{\psi}_p\) are
pollutant-specific. The hourly influence \(h\) on \(\lambda_{sthp}\) is
solely represented through the hourly indicator variables in
\(\symbf{X}_{th}\). This approach offers computational advantages by
reducing the dimensionality of the random effects.

In addition to the fixed effects, \(\lambda_{sthp}\) includes three
random effects that are multivariate with four dimensions, corresponding
to each pollutant:

\begin{itemize}
    \tightlist
    \item
    $\symbf{Q}_s=(Q_{s1},\dots,Q_{sP})^{\top}$, a station-level random effect with covariance matrix
    $\Omega$;
    \item
    $\symbf{D}_t=(D_{t1},\dots,D_{tP})^{\top}$, a first order vector autoregressive time trend with a
    diagonal transition matrix $\mbox{diag}(\symbf{\phi})$ and
    seasonally varying covariance $\symbf{\Sigma}_t$; 
    \item
    $\symbf{C}_{st}=(C_{st1},\dots,C_{stP})^{\top}$, temporally independent station-time interaction
    effects, with a time-varying covariance matrix which is a rescaling of
    $\symbf{\Sigma}_t$.
\end{itemize}

The correlation between O\(_3\) and the other three pollutants exhibits
seasonal variability, with a general positive correlation during summer
and a negative correlation during winter \autocites[see
e.g.,][]{Chen2019,ChenK2017,Huang2022}. The four pollutants,
PM\(_{2.5}\), PM\(_{10}\), NO\(_2\), and O\(_3\), are arranged in a
specific order. The last row and column of the covariance matrix
\(\symbf{\Sigma}_t\) represent the time-varying covariances between
O\(_3\) and the other pollutants.

To model the correlation structure among the first three pollutants
(PM\(_{2.5}\), PM\(_{10}\), and NO\(_2\)), we utilize a three-by-three
lower-triangular matrix \(\symbf{\bar{L}}\). This matrix is derived from
the Cholesky decomposition of the correlation matrix and involves three
parameters that need to be estimated. The scalar-valued variable
\(L_{tp}\) is influenced by intercept parameters \(b_p\) and four
coefficients \(\symbf{\eta}_p\). These coefficients are associated with
sine and cosine functions with periods of 6 and 12 months, which are
enclosed within the time-varying factor \(\symbf{f}_t\). The specific
form of \(L_{tp}\) depends on these parameters and captures the
relationship between the pollutant concentrations.

In summary, the model accounts for the seasonal variation in the
correlation between O\(_3\) and the other pollutants by incorporating a
time-varying covariance matrix \(\symbf{\Sigma}_t\) which includes a
lower-triangular matrix \(\symbf{\bar{L}}\) representing the correlation
structure among the first three pollutants (PM\(_{2.5}\), PM\(_{10}\)
and NO\(_2\)), and scalar-valued variables \(L_{tp}\). As the model is
under a Bayesian framework, prior distributions are required for the
model parameters, and we employ the same priors as used in
\textcite{Huang2022}.

\hypertarget{daily-pollution-model}{%
\subsection{Daily pollution model}\label{daily-pollution-model}}

For the daily pollution model, the response variable
\(Y_{tp}=\frac{\sum_s\sum_h Y_{sthp}}{\mid s \mid \cdot \mid h \mid}\)
represents the daily average concentration of pollutant \(p\) on day
\(t\). It is assumed to follow a Gamma distribution with a shape
parameter \(\alpha_p\) and rate parameter \(\alpha_p/\lambda_{tp}\).

The log of the mean concentration, \(\log(\lambda_{tp})\), is modelled
as a linear combination of covariates represented by \(\symbf{X}_t\) and
an additional term \(D_{tp}\). The covariates \(\symbf{X}_t\) consist of
sinusoidal functions that capture seasonality at frequencies of 6 and 12
months, day-of-week indicators, and a linear time trend. The fixed
effect coefficients \(\symbf{\psi}_p\) are pollutant-specific.

The term \(\symbf{D}_t=(D_{t1},\dots,D_{tP})^{\top}\) represents a
random effect component that captures temporal dependencies. It follows
a multivariate normal distribution with a mean determined by the
previous day's value \(\symbf{D}_{t-1}\) multiplied by the diagonal
transition matrix \(\text{diag}(\symbf{\phi})\), and a covariance matrix
\(\symbf{\Sigma}_t\). The covariance matrix \(\symbf{\Sigma}_t\) is
constructed the same as in the aforementioned hourly pollution model,
accounting for the seasonal variation in the correlation between O\(_3\)
and the other pollutants. \begin{equation}\label{eqnPollutiondaily}
\begin{aligned}
   Y_{tp} \sim & \mbox{Gamma}
   \left( \alpha_p, \alpha_p / \lambda_{tp} \right) \\
  \log{(\lambda_{tp})}  = & \symbf{X}_{t}^{\top}\symbf{\psi}_{p} + D_{tp} \\
    \symbf{D}_{t}  \mid \symbf{D}_{t-1}   \sim & \mbox{N}\left[\mbox{diag}(\symbf{\phi})  \symbf{D}_{t-1} ,\ \symbf{\Sigma}_t \right]\\
\symbf{\Sigma}_t  = &  \mbox{diag}(\symbf{\delta}) \left[
\begin{array}{cccc}
 &  &  &  \\
 & \symbf{\bar{L}}  &   &  {\symbf{0}}  \\ 
  &  &  &  \\
 L_{t1} & \dots & L_{t,P-1}  &  L_{tP}
\end{array}\right] \left[
\begin{array}{cccc}
 &  &  &  L_{t1} \\
 & \symbf{\bar{L}}^{\top}  &   &  \dots  \\ 
  &  &  & L_{t,P-1} \\
  & {\symbf{0}}^{\top} &   &  L_{tP}
\end{array}\right] \mbox{diag}(\symbf{\delta}) \\
L_{tp}  = &  b_p + \symbf{f}_t ^{\top}\symbf{\eta}_p
  \end{aligned}
\end{equation}

Since the model operates within a Bayesian framework, it necessitates
the specification of prior distributions for the model parameters. To
ensure consistency, we adopt the same priors that were utilized in the
previously mentioned hourly pollution model.

\hypertarget{study-region}{%
\subsection{Study region}\label{study-region}}

The study region is Guangzhou, also known as Canton, which is a vibrant
metropolis located in southern China. There are total 10 pollution
monitoring stations across Guangzhou (see Figure \ref{fig:guangzhou}).
Hourly pollution data from 2017 to 2022 were collected from China
National Environmental Monitoring Centre
(\url{http://www.cnemc.cn/sssj/}), including PM\(_{2.5}\), PM\(_{10}\),
NO\(_2\), and O\(_3\).

Table \ref{tab:datasummary2} presents a comprehensive summary of the
mean concentrations and missing data percentages for four pollutants
(PM\(_{2.5}\), PM\(_{10}\), NO\(_2\), and O\(_3\)) across various
stations. In the table, NO\(_2\) shows moderate to high mean
concentrations across all stations, ranging from 22.16 to 48.76
\(\mu g/m^{3}\). O\(_3\) exhibits relatively high mean concentrations,
with values ranging from 49.37 to 63.30 \(\mu g/m^{3}\). PM\(_{10}\) and
PM\(_{2.5}\) have lower mean concentrations compared to O\(_3\), varying
from 37.19 to 55.07 \(\mu g/m^{3}\) for PM\(_{10}\) and from 24.24 to
30.98 \(\mu g/m^{3}\) for PM\(_{2.5}\). The missing data percentages for
all pollutants are generally low. PM\(_{10}\) has the highest missing
percentage among the pollutants, with values ranging from 3.32\% to
9.75\%. PM\(_{2.5}\) has relatively lower missing percentages, ranging
from 1.06\% to 2.85\%.

\begin{figure}
  \centering
  \includegraphics[width=1\textwidth]{stationMap.png}
  \caption{Pollutant concentrations in each monitoring station in Guangzhou, with the mean concentrations across stations being 49, 28, 40, 54 $\mu g/m^{3}$ for PM$_{10}$, PM$_{2.5}$, NO$_2$ and O$_3$, respectively.}
  \label{fig:guangzhou} 
\end{figure}

\begin{table}[h]

\caption{\label{tab:datasummary2}The mean concentrations ($\mu g/m^{3}$) and missing percentage (\%) by pollutant and station.}
\centering
\begin{tabular}[t]{lrrrrrrrr}\hline
\multicolumn{1}{c}{\textbf{ }} & \multicolumn{4}{c}{\textbf{Mean concentrations}} & \multicolumn{4}{c}{\textbf{Missing percentage}} \\\hline
Station & NO$_2$ & O$_3$ & PM$_{10}$ & PM$_{2.5}$ & NO$_2$ & O$_3$ & PM$_{10}$ & PM$_{2.5}$\\\hline
1345A & 48.76 & 49.46 & 55.07 & 30.98 & 1.63 & 1.46 & 4.44 & 1.06\\
1346A & 41.05 & 55.46 & 50.08 & 30.18 & 2.53 & 2.43 & 5.63 & 1.91\\
1348A & 43.85 & 53.06 & 48.11 & 27.65 & 2.67 & 2.88 & 5.14 & 2.11\\
1349A & 48.10 & 49.53 & 53.15 & 27.64 & 1.89 & 2.56 & 3.32 & 1.71\\
1350A & 39.02 & 59.11 & 48.22 & 28.16 & 2.10 & 2.62 & 6.37 & 1.68\\
1351A & 36.68 & 60.96 & 49.19 & 28.14 & 1.83 & 2.35 & 4.54 & 1.31\\
1352A & 45.99 & 53.94 & 51.28 & 28.46 & 1.82 & 1.77 & 4.16 & 1.23\\
1353A & 30.64 & 50.62 & 49.66 & 27.40 & 2.50 & 5.10 & 6.09 & 2.05\\
1354A & 41.34 & 49.37 & 45.44 & 26.99 & 2.52 & 2.02 & 5.31 & 1.54\\
1355A & 22.16 & 63.30 & 37.19 & 24.24 & 3.51 & 3.43 & 9.75 & 2.85\\\hline
\end{tabular}
\end{table}

\hypertarget{study-design}{%
\subsection{Study design}\label{study-design}}

To compare the performance of the hourly pollution model and daily
pollution model, we divided the hourly dataset into two subsets: a
training dataset used for model fitting and a validation dataset for
evaluating model performance.

The validation dataset was specifically designed to assess the model's
performance under different missing data scenarios. We considered two
distinct scenarios: (1) one-pollutant missing across all stations for 50
random days, and (2) two-pollutants missing across all stations for 50
random days. The dataset covered four pollutants: PM\(_{2.5}\),
PM\(_{10}\), NO\(_2\), and O\(_3\). For the first scenario, we examined
four distinct cases: (a) PM\(_{2.5}\) missing, (b) PM\(_{10}\) missing,
(c) NO\(_2\) missing, and (d) O\(_3\) missing. In the second scenario,
where two pollutants were missing across all stations, we analyzed six
different cases, including (a) both PM\(_{2.5}\) and PM\(_{10}\)
missing, (b) both PM\(_{2.5}\) and NO\(_2\) missing, and so on. Be aware
that the simulated missing data scenarios intended for evaluation
purposes will be incorporated into the pre-existing actual instances of
missing data within the original dataset. This inclusion will contribute
to the overall percentage of missing data used in fitting the models. As
a result, the model evaluation was carried out under a worst-case
scenario, involving a marginally higher proportion of missing data
compared to what is present in the genuine dataset. By doing so, we can
demonstrate the model's ability to handle more severe missing data
conditions and still provide good performance in imputing missing
values. These diverse scenarios enable us to conduct a comprehensive
evaluation of the model's performance under various missing data
situations.

\hypertarget{results}{%
\section{Results}\label{results}}

We employed the Stan software \autocite{stan} to obtain posterior
samples from both the hourly and daily air pollution models, which has a
user-friendly coding environment, rendering it a favoured choice among
researchers. Stan's Hamiltonian Monte Carlo (HMC) algorithm demonstrates
strong performance in terms of chain mixing and convergence. However,
due to the algorithm's complexity, Stan can be computationally
intensive. This becomes a particular concern in our application, as the
dataset comprises a large number of hourly observations for four
pollutants from multiple monitoring stations. In contrast, the proposed
daily pollution model offers a substantial advantage in terms of
computing time. Recognizing this, our study is motivated by the idea
that if the daily pollution model provides sufficient accuracy, it would
be preferable to utilize it instead of the more computationally
demanding hourly pollution model. In our study, the time required to run
a chain with 1000 iterations for hourly pollution models is about 3-4
days, while it takes only about 1-2 hours for daily pollution models,
using Intel(R) Core(TM) i9-9900K CPU @ 3.60GHz.

Two parallel chains are run for each scenario case with each chain
having 1000 iterations. The first 500 iterations are removed as the
burn-in period, after which convergence has been assessed and confirmed
by the \texttt{monitor} function in Stan.

\hypertarget{model-parameter-estimation}{%
\subsection{Model parameter
estimation}\label{model-parameter-estimation}}

The main results from fitting pollution models, including the estimates
of unknown parameters, and the posterior samples of seasonally varying
correlations between pollutants, are presented in Table
\ref{tab:pollutionModelImplementationboth} and Figure
\ref{fig:varyingcov}, respectively. Overall, Table
\ref{tab:pollutionModelImplementationboth} provides a comparison between
the hourly and daily pollution models by presenting the posterior
medians and 95\% credible intervals for various parameters. It is
observed from Table \ref{tab:pollutionModelImplementationboth} that the
temporal random effects \(D_{tp}\) and their corresponding
autoregressive coefficient \(\phi_p\) exhibit similarities in both the
hourly and daily pollution models. This similarity can be attributed to
the fact that these effects are designed based on a daily unit rather
than an hourly unit, emphasizing the association with days rather than
individual hours. As expected, the coefficient of variation
\(1/\sqrt\alpha_p\) is much smaller from the daily pollution model,
since daily averages tend to have lower dispersion than hourly data.
Moving to the second part of Table
\ref{tab:pollutionModelImplementationboth}, which focuses on the hourly
pollution model specifically. The estimates for \(\kappa_p\) indicate
the variability in the random intercepts for each station \(s\) and
pollutant \(p\). We observe that the estimates for \(\kappa_p\) vary
across pollutants, with PM\(_{2.5}\) and PM\(_{10}\) having lower
variability than NO\(_2\) and O\(_3\). Similar results were observed
from the random intercepts \(\gamma_p\) for each station \(s\) and
pollutant \(p\) on day \(t\). Regarding the contamination measure
\(\rho_1\) and \(\rho_2\), both of them suggest that the contamination
rate of the data is relatively low in our study.

Posterior samples of correlations among pollutants for both hourly and
daily pollution models are shown in Figure \ref{fig:varyingcov}, and the
correlations between PM\(_{2.5}\) and O\(_3\), between NO\(_2\) and
O\(_3\) for the temporal random walk effect, \(\symbf{D}_t\), are
different in summer and winter (the similar correlations between
PM\(_{10}\) and O\(_3\) are not shown here). This speaks to the need for
seasonally varying correlations in the air pollution model.

\begin{table}[!tbp]
{\small
\caption{Posterior medians and 95\% credible intervals for parameters in both hourly and daily pollution models.\label{tab:pollutionModelImplementationboth}} 
\begin{center}
\begin{tabular}{lccccccc}
\hline\hline
\multicolumn{1}{l}{\bfseries }&\multicolumn{3}{c}{\bfseries Hourly model}&\multicolumn{1}{c}{\bfseries }&\multicolumn{3}{c}{\bfseries Daily model}\tabularnewline
\cline{2-4} \cline{6-8}
\multicolumn{1}{l}{}&\multicolumn{1}{c}{Est}&\multicolumn{1}{c}{2.5\%}&\multicolumn{1}{c}{97.5\%}&\multicolumn{1}{c}{}&\multicolumn{1}{c}{Est}&\multicolumn{1}{c}{2.5\%}&\multicolumn{1}{c}{97.5\%}\tabularnewline
\hline
\multicolumn{8}{l}{\bfseries Temporal: $\delta_p\propto\mbox{sd}(D_{tp})$}\tabularnewline\hline
~~PM$_{2.5}$&$0.314$&$0.307$&$0.322$&&$0.366$&$0.354$&$0.377$\tabularnewline
~~PM$_{10}$&$0.307$&$0.300$&$0.315$&&$0.346$&$0.335$&$0.356$\tabularnewline
~~NO$_2$&$0.266$&$0.258$&$0.275$&&$0.250$&$0.242$&$0.258$\tabularnewline
~~O$_3$&$0.910$&$0.640$&$1.307$&&$0.811$&$0.583$&$1.142$\tabularnewline
\hline
\multicolumn{8}{l}{\bfseries AR coefficient: $\phi_p$}\tabularnewline\hline
~~PM$_{2.5}$&$0.734$&$0.714$&$0.755$&&$0.690$&$0.670$&$0.708$\tabularnewline
~~PM$_{10}$&$0.734$&$0.713$&$0.753$&&$0.704$&$0.685$&$0.722$\tabularnewline
~~NO$_2$&$0.756$&$0.733$&$0.777$&&$0.711$&$0.689$&$0.734$\tabularnewline
~~O$_3$&$0.806$&$0.781$&$0.829$&&$0.746$&$0.719$&$0.769$\tabularnewline
\hline
\multicolumn{8}{l}{\bfseries Observations: $1/\sqrt\alpha_p=\mbox{sd}(Y_{tp})/\mbox{E}(Y_{tp})$}\tabularnewline\hline
~~PM$_{2.5}$&$0.324$&$0.324$&$0.325$&&$0.015$&$0.010$&$0.025$\tabularnewline
~~PM$_{10}$&$0.316$&$0.315$&$0.317$&&$0.013$&$0.008$&$0.018$\tabularnewline
~~NO$_2$&$0.324$&$0.324$&$0.325$&&$0.022$&$0.015$&$0.036$\tabularnewline
~~O$_3$&$0.631$&$0.630$&$0.633$&&$0.203$&$0.187$&$0.218$\tabularnewline
\hline
\multicolumn{8}{c}{\bfseries Hourly model}\tabularnewline 
\hline
\multicolumn{1}{l}{}&\multicolumn{1}{c}{Est}&\multicolumn{1}{c}{2.5\%}&\multicolumn{1}{c}{97.5\%}&\multicolumn{1}{c}{}&\multicolumn{1}{c}{Est}&\multicolumn{1}{c}{2.5\%}&\multicolumn{1}{c}{97.5\%}\tabularnewline\hline
\multicolumn{4}{l}{\bfseries Station-level: $\kappa_p=\mbox{sd}(Q_{sp})$}&&\multicolumn{3}{l}{\bfseries Contamination $\rho_1$(\textperthousand)}\tabularnewline\hline
~~PM$_{2.5}$&$0.070$&$0.047$&$0.111$&&$0.707$&$0.603$&$0.827$\tabularnewline
~~PM$_{10}$&$0.090$&$0.063$&$0.148$&&$0.093$&$0.055$&$0.138$\tabularnewline
~~NO$_2$&$0.248$&$0.168$&$0.410$&&$0.248$&$0.188$&$0.314$\tabularnewline
~~O$_3$&$0.168$&$0.105$&$0.291$&&$0.006$&$0.000$&$0.032$\tabularnewline
\hline
\multicolumn{4}{l}{\bfseries Daily/Station: $\gamma_p\propto\mbox{sd}(C_{tsp})$}&&\multicolumn{3}{l}{\bfseries Contamination $\rho_2$(\textperthousand)}\tabularnewline\hline
~~PM$_{2.5}$&$0.069$&$0.067$&$0.072$&&$2.933$&$2.727$&$3.153$\tabularnewline
~~PM$_{10}$&$0.074$&$0.071$&$0.076$&&$0.986$&$0.866$&$1.116$\tabularnewline
~~NO$_2$&$0.195$&$0.188$&$0.202$&&$0.044$&$0.022$&$0.078$\tabularnewline
~~O$_3$&$0.185$&$0.132$&$0.269$&&$1.203$&$0.973$&$1.431$\tabularnewline
\hline
\end{tabular}\end{center}}
\end{table}

\begin{figure}[H]

{\centering \subfloat[cor(${D}_{t,PM_{2.5}}, {D}_{t,O_{3}}$)\label{fig:varyingcov-1}]{\includegraphics[width=0.45\textwidth]{figure/varyingcov-1} }\subfloat[cor(${D}_{t,NO_{2}}, {D}_{t,O_{3}}$)\label{fig:varyingcov-2}]{\includegraphics[width=0.45\textwidth]{figure/varyingcov-2} }

}

\caption{ Posterior samples of seasonally varying correlations between O$_3$ and each of PM$_{2.5}$ and NO$_2$, from hourly pollution model (black) and daily pollution model (blue).}\label{fig:varyingcov}
\end{figure}

\hypertarget{model-prediction-performance}{%
\subsection{Model prediction
performance}\label{model-prediction-performance}}

In this section, we analyze and compare the prediction performance of
the hourly and daily pollution models under different scenarios,
specifically considering the presence of missing data for one pollutant
or two pollutants. Recall that in the first scenario, we evaluate the
models' performance when one pollutant is missing across all stations.
We consider four distinct cases: PM\(_{2.5}\) missing, PM\(_{10}\)
missing, NO\(_2\) missing, and O\(_3\) missing. In the second scenario,
we explore the models' performance when two pollutants are missing
across all stations. This scenario encompasses six different cases, such
as both PM\(_{2.5}\) and PM\(_{10}\) missing, both PM\(_{2.5}\) and
NO\(_2\) missing, and so on.

\begin{table}

\caption{Prediction performance comparison between hourly, daily and baseline pollution models in terms of bias and root mean square error ($\mu g/m^{3}$).}\label{tab:hourlydailycomparison}
\centering
\begin{tabular}[h]{lllllllll}
\hline\hline
\multicolumn{1}{c}{\textbf{ }} & \multicolumn{2}{c}{\textbf{Hourly model}} & \multicolumn{2}{c}{\textbf{Daily model}} & \multicolumn{2}{c}{\textbf{B-model-1}} & \multicolumn{2}{c}{\textbf{B-model-2}} \\\hline
Pollutant  & Bias & RMSE & Bias & RMSE & Bias & RMSE & Bias & RMSE\\\hline
\multicolumn{5}{l}{\textbf{One-pollutant missing}}\\\hline
\hspace{1em}PM$_{2.5}$ & 0.53 & 5.85 & 0.38 & 4.86 & 2.78 & 10.59 & -2.82 & 14.72\\
\hspace{1em}PM$_{10}$ & 0.36 & 7.26 & -0.38 & 5.45 & 1.4 & 22.29 & 3.69 & 35.36\\
\hspace{1em}NO$_{2}$  & -2.48 & 13.07 & -2.34 & 12.72 & 1.14 & 10.33 & -4 & 30.46\\
\hspace{1em}O$_{3}$   & 5.83 & 30.9 & 1.57 & 24.06 & -1.78 & 18.38 & -3.19 & 29.55\\\hline
\multicolumn{5}{l}{\textbf{Two-pollutant missing}}\\\hline
\hspace{1em}PM$_{2.5}$ & 1.33 & 8.77 & 1.16 & 7.99 & 0.63 & 9.1 & 1.25 & 14.66\\
\hspace{1em}PM$_{10}$ & 1.62 & 13.59 & 1.09 & 12.37 & 0.4 & 14.49 & -0.26 & 26.67\\
\hspace{1em}PM$_{2.5}$ & -0.61 & 6.37 & -1.27 & 5.12 & 8.34 & 13.59 & -3.94 & 22.06\\
\hspace{1em}NO$_{2}$  & 0.21 & 9.28 & 1.03 & 9.03 & 3.94 & 11.88 & 1 & 17.59\\
\hspace{1em}PM$_{2.5}$ & -0.19 & 4.51 & -0.33 & 3.74 & 1.41 & 11.47 & 6.56 & 21.56\\
\hspace{1em}O$_{3}$   & 5.75 & 33.68 & -2.38 & 22.14 & -5.83 & 26.78 & 8.81 & 45.49\\
\hspace{1em}PM$_{10}$ & 0.03 & 9.7 & -0.09 & 6.71 & -2.2 & 9.8 & 1.17 & 29.46\\
\hspace{1em}NO$_{2}$  & -1.77 & 12.08 & -1.44 & 11.71 & -2.53 & 9.83 & -2.35 & 12.16\\
\hspace{1em}PM$_{10}$ & -0.19 & 8.16 & -0.98 & 5.99 & -7.65 & 20.17 & -10.34 & 28.74\\
\hspace{1em}O$_{3}$   & 8.28 & 38.63 & 0.64 & 24.23 & -4.21 & 18.25 & -1.83 & 29.27\\
\hspace{1em}NO$_{2}$  & 1.3 & 10.82 & 0.84 & 8.78 & 1.29 & 9.08 & 7.74 & 20.95\\
\hspace{1em}O$_{3}$  & 11.67 & 37.02 & 0.49 & 22.48 & 2.84 & 24.66 & 18.45 & 40.27\\\hline
\end{tabular}
\end{table}

In order to assess the efficacy of our advanced proposed models, we
conduct a comparative analysis that includes two baseline models. The
initial baseline, referred to as ``B-model-1'', employs the last daily
observations for predictions. Meanwhile, the second baseline, denoted as
``B-model-2,'' incorporates observations from the same weekdays in the
last preceding weeks for predictions. Table
\ref{tab:hourlydailycomparison} presents the prediction performances
from hourly, daily and baseline pollution models. The primary insight
derived from Table \ref{tab:hourlydailycomparison} is that both the
hourly and daily models offer substantial advantages over the basic
baseline models. This is evident from the generally notable reduction in
bias and root mean square error (RMSE) observed in the results. The
considerable enhancements in these pivotal performance metrics
underscore the distinct advantages inherent in our approach.

Regarding the performance comparison between the hourly and daily
models, when considering scenarios with a single-pollutant missing, it
becomes evident that the hourly pollution model tends to display
elevated levels of bias and RMSE in contrast to the daily pollution
model. For PM\(_{2.5}\), both models show relatively low bias, but the
hourly pollution model has a slightly higher RMSE than the daily
pollution model. Similarly, for PM\(_{10}\), the hourly pollution model
has a higher RMSE compared to the daily pollution model. Furthermore,
for NO\(_2\), the prediction performance is similar for both models, but
for O\(_3\), the hourly pollution model displays significantly higher
bias and RMSE compared to the daily pollution model.

In the second part of Table \ref{tab:hourlydailycomparison}, which
represents the two-pollutant missing scenario, a few main findings can
be observed. In general, the daily pollution models outperform the
hourly pollution models in terms of bias and RMSE. The daily pollution
models generally consistently exhibit lower bias and RMSE values
compared to the hourly pollution models, indicating their superior
prediction performance. When both PM\(_{2.5}\) and PM\(_{10}\) are
missing, their prediction performance is not as good as in the
one-pollutant missing scenario. This outcome can be attributed to the
high correlation between PM\(_{2.5}\) and PM\(_{10}\). Since these two
pollutants are strongly related, having information about one pollutant
greatly aids in predicting the other. Thus, when both are missing, the
models face increased difficulty in accurately predicting their
concentrations. O\(_3\), being a challenging pollutant to predict
precisely, shows higher bias and RMSE values overall. Despite the
inherent difficulty in predicting O\(_3\) concentrations accurately, the
daily pollution models demonstrate a noticeable improvement over the
hourly pollution models, indicating their effectiveness in capturing
O\(_3\) patterns and trends.

Figure \ref{fig:singleMissing} displays the boxplot of posterior samples
of the predicted daily pollution concentrations in four distinct cases
of single-pollutant missing scenarios: PM\(_{2.5}\) missing, PM\(_{10}\)
missing, NO\(_2\) missing, and O\(_3\) missing. The predictions are
generated by both the hourly and daily pollution models, with the true
daily mean indicated in red. Firstly, it is worth noting that both the
hourly and daily pollution models exhibit good prediction performance.
Despite the missing pollutant in each case, the models effectively
capture the underlying patterns and trends of the pollution
concentrations. This observation suggests that the models are able to
generate reliable predictions for the missing pollutants. Secondly,
Figure \ref{fig:singleMissing} clearly illustrates that the daily
pollution model generally consistently outperforms the hourly pollution
model. This is evident from the narrower boxplots in the daily pollution
model, indicating a reduced spread of predicted values and ultimately
resulting in lower RMSE. The improved performance of the daily pollution
model compared to the hourly pollution model suggests its ability to
provide more accurate and precise predictions regarding daily
concentrations.

Similarly, Figure \ref{fig:pm25pm10missing} displays the boxplot of
posterior samples of the predicted daily pollution concentrations in a
specific case of two-pollutant missing scenario: both PM\(_{2.5}\) and
PM\(_{10}\) missing (the results of other cases of two-pollutant missing
scenario are presented in Appendix to save space). The figure also
demonstrates the superior performance of the daily pollution model over
the hourly pollution model in terms of reduced spread and lower RMSE.
Additionally, it highlights the capability of both models to accurately
predict pollution concentrations when a few observations are missing
from the input data.

\begin{figure}[H]
\subfloat[PM$_{2.5}$ missing, Hourly model\label{fig:singleMissing-1}]{\includegraphics[width=0.49\textwidth]{figure/singleMissing-1} }\subfloat[PM$_{2.5}$ missing, Daily model\label{fig:singleMissing-2}]{\includegraphics[width=0.49\textwidth]{figure/singleMissing-2} }\subfloat[PM$_{10}$ missing, Hourly model\label{fig:singleMissing-3}]{\includegraphics[width=0.49\textwidth]{figure/singleMissing-3} }\subfloat[PM$_{10}$ missing, Daily model\label{fig:singleMissing-4}]{\includegraphics[width=0.49\textwidth]{figure/singleMissing-4} }\subfloat[NO$_{2}$ missing, Hourly model\label{fig:singleMissing-5}]{\includegraphics[width=0.49\textwidth]{figure/singleMissing-5} }\subfloat[NO$_{2}$ missing, Daily model\label{fig:singleMissing-6}]{\includegraphics[width=0.49\textwidth]{figure/singleMissing-6} }\subfloat[O$_{3}$ missing, Hourly model\label{fig:singleMissing-7}]{\includegraphics[width=0.49\textwidth]{figure/singleMissing-7} }\subfloat[O$_{3}$ missing, Daily model\label{fig:singleMissing-8}]{\includegraphics[width=0.49\textwidth]{figure/singleMissing-8} }\caption[Boxplot of posterior samples of the predicted daily pollution concentrations ($\mu g/m^3$) under four single-pollutant missing scenarios from hourly and daily pollution models, with outliers denoted as blue crosses and the true daily mean being in red dots]{Boxplot of posterior samples of the predicted daily pollution concentrations ($\mu g/m^3$) under four single-pollutant missing scenarios from hourly and daily pollution models, with outliers denoted as blue crosses and the true daily mean being in red dots.}\label{fig:singleMissing}
\end{figure}

\begin{figure}[H]
\subfloat[Hourly model\label{fig:pm25pm10missing-1}]{\includegraphics[width=0.49\textwidth]{figure/pm25pm10missing-1} }\subfloat[Daily model\label{fig:pm25pm10missing-2}]{\includegraphics[width=0.49\textwidth]{figure/pm25pm10missing-2} }\subfloat[Hourly model\label{fig:pm25pm10missing-3}]{\includegraphics[width=0.49\textwidth]{figure/pm25pm10missing-3} }\subfloat[Daily model\label{fig:pm25pm10missing-4}]{\includegraphics[width=0.49\textwidth]{figure/pm25pm10missing-4} }\caption{Boxplot of posterior samples of the predicted daily PM$_{2.5}$ ($\mu g/m^3$) and PM$_{10}$ ($\mu g/m^3$) under two-pollutant missing scenario (PM$_{2.5}$ and PM$_{10}$) from hourly and daily pollution models, with outliers denoted as blue crosses and the true daily mean being in red dots.}\label{fig:pm25pm10missing}
\end{figure}

\hypertarget{discussion}{%
\section{Discussion}\label{discussion}}

When investigating the short-term health impacts of air pollution, the
estimation of daily exposure plays a fundamental role. In our study, we
embarked on a comparative analysis between hourly and daily pollution
modelling strategies, aiming to evaluate their efficacy in the context
of estimating daily pollution exposure. We posited that the daily
pollution modelling approach could yield comparable outcomes to its
hourly counterpart, while simultaneously mitigating model intricacy and
computation demands.

Utilizing the dataset sourced from Guangzhou city, we constructed a
comprehensive array of missing-data scenarios and then applied these
scenarios to assess the efficacy of both hourly and daily pollution
models. Our results clearly demonstrate that the daily pollution models
consistently exhibit lower bias and RMSE values compared to the hourly
pollution models, indicating their superior ability to capture and
predict pollutant concentrations. These findings support the hypothesis
that daily pollution modelling can provide comparable results to hourly
pollution modelling while offering computational advantages. These
findings can guide researchers in selecting the most suitable modelling
approach for estimating pollution exposure in epidemiological studies.

Why the daily data modelling outperforms the hourly data modelling?
There are several probable explanations for why directly predicting
daily pollution concentrations tends to be more accurate than predicting
hourly concentrations and then aggregating them. Firstly, direct daily
predictions help to reduce noise and variability inherent in hourly
data, leading to a more stable and accurate estimate. Additionally,
aggregating hourly data involves an additional step of calculation,
which can introduce errors or inaccuracies in the process. Predicting
daily concentrations directly simplifies the modelling approach,
potentially leading to a more accurate estimation of daily exposure.
Furthermore, temporal patterns in pollution levels, like daily routines
and atmospheric conditions, are likely better captured in direct daily
predictions, enhancing accuracy. In practical terms, researchers might
incorporate both hourly and daily models within a study to achieve a
more accurate estimation of daily pollution exposure. The hourly model
could be employed to fill in the missing hourly values on days with only
a few instances of missing hourly data. Conversely, the daily model
would be better suited to address days characterized by a significant
amount of missing hourly data or even a complete absence of
observations.

We make the code and data openly accessible, so other researchers and
interested individuals can examine the methodology, replicate the
results, and build upon the findings. This promotes the advancement of
scientific knowledge and fosters a collaborative research environment.
Additionally, the availability of the code and data encourages the
adoption of best practices, facilitates benchmarking against existing
methods, and supports the development of novel approaches in the field.

\hypertarget{data-availability}{%
\section*{Data availability}\label{data-availability}}
\addcontentsline{toc}{section}{Data availability}

The code and data used for this study are publicly shared on GitHub
(\url{https://github.com/hgw0610209/hourlydailymodelcomparison}),
allowing for transparency, reproducibility, and collaboration.

\hypertarget{acknowledgements}{%
\section*{Acknowledgements}\label{acknowledgements}}
\addcontentsline{toc}{section}{Acknowledgements}

We would like to thank the editor, and two anonymous referees for their
insightful and constructive comments, which greatly improved the
presentation of the article. Dr.~Huang is funded by Shantou University
funding NTF21002.

\hypertarget{appendix}{%
\section*{Appendix}\label{appendix}}
\addcontentsline{toc}{section}{Appendix}

The predicted daily pollution concentrations in other cases of
two-pollutant missing scenario are presented in Figure
\ref{fig:pm25no2missing}, \ref{fig:pm25o3missing},
\ref{fig:pm10no2missing}, \ref{fig:pm10o3missing},
\ref{fig:no2o3missing}. The figures demonstrate the superior performance
of the daily pollution model over the hourly pollution model in terms of
reduced spread and lower RMSE.

\begin{figure}[H]
\subfloat[Hourly model\label{fig:pm25no2missing-1}]{\includegraphics[width=0.49\textwidth]{figure/pm25no2missing-1} }\subfloat[Daily model\label{fig:pm25no2missing-2}]{\includegraphics[width=0.49\textwidth]{figure/pm25no2missing-2} }\subfloat[Hourly model\label{fig:pm25no2missing-3}]{\includegraphics[width=0.49\textwidth]{figure/pm25no2missing-3} }\subfloat[Daily model\label{fig:pm25no2missing-4}]{\includegraphics[width=0.49\textwidth]{figure/pm25no2missing-4} }\caption{Boxplot of posterior samples of the predicted daily PM$_{2.5}$ ($\mu g/m^3$) and NO$_{2}$ ($\mu g/m^3$) under two-pollutant missing scenario (PM$_{2.5}$ and NO$_2$) from hourly and daily pollution models, with outliers denoted as blue crosses and the true daily mean being in red dots.}\label{fig:pm25no2missing}
\end{figure}

\begin{figure}[H]
\subfloat[Hourly model\label{fig:pm25o3missing-1}]{\includegraphics[width=0.49\textwidth]{figure/pm25o3missing-1} }\subfloat[Daily model\label{fig:pm25o3missing-2}]{\includegraphics[width=0.49\textwidth]{figure/pm25o3missing-2} }\subfloat[Hourly model\label{fig:pm25o3missing-3}]{\includegraphics[width=0.49\textwidth]{figure/pm25o3missing-3} }\subfloat[Daily model\label{fig:pm25o3missing-4}]{\includegraphics[width=0.49\textwidth]{figure/pm25o3missing-4} }\caption{Boxplot of posterior samples of the predicted daily PM$_{2.5}$ ($\mu g/m^3$) and O$_{3}$ ($\mu g/m^3$) under two-pollutant missing scenario (PM$_{2.5}$ and O$_3$) from hourly and daily pollution models, with outliers denoted as blue crosses and the true daily mean being in red dots.}\label{fig:pm25o3missing}
\end{figure}

\begin{figure}[H]
\subfloat[Hourly model\label{fig:pm10no2missing-1}]{\includegraphics[width=0.49\textwidth]{figure/pm10no2missing-1} }\subfloat[Daily model\label{fig:pm10no2missing-2}]{\includegraphics[width=0.49\textwidth]{figure/pm10no2missing-2} }\subfloat[Hourly model\label{fig:pm10no2missing-3}]{\includegraphics[width=0.49\textwidth]{figure/pm10no2missing-3} }\subfloat[Daily model\label{fig:pm10no2missing-4}]{\includegraphics[width=0.49\textwidth]{figure/pm10no2missing-4} }\caption[Boxplot of posterior samples of the predicted daily PM$_{10}$ ($\mu g/m^3$) and NO$_{2}$ ($\mu g/m^3$) under two-pollutant missing scenario (PM$_{10}$ and NO$_2$) from hourly and daily pollution models, with outliers denoted as blue crosses and the true daily mean being in red dots]{Boxplot of posterior samples of the predicted daily PM$_{10}$ ($\mu g/m^3$) and NO$_{2}$ ($\mu g/m^3$) under two-pollutant missing scenario (PM$_{10}$ and NO$_2$) from hourly and daily pollution models, with outliers denoted as blue crosses and the true daily mean being in red dots.}\label{fig:pm10no2missing}
\end{figure}

\begin{figure}[H]
\subfloat[Hourly model\label{fig:pm10o3missing-1}]{\includegraphics[width=0.49\textwidth]{figure/pm10o3missing-1} }\subfloat[Daily model\label{fig:pm10o3missing-2}]{\includegraphics[width=0.49\textwidth]{figure/pm10o3missing-2} }\subfloat[Hourly model\label{fig:pm10o3missing-3}]{\includegraphics[width=0.49\textwidth]{figure/pm10o3missing-3} }\subfloat[Daily model\label{fig:pm10o3missing-4}]{\includegraphics[width=0.49\textwidth]{figure/pm10o3missing-4} }\caption[Boxplot of posterior samples of the predicted daily PM$_{10}$ ($\mu g/m^3$) and O$_{3}$ ($\mu g/m^3$) under two-pollutant missing scenario (PM$_{10}$ and O$_3$) from hourly and daily pollution models, with outliers denoted as blue crosses and the true daily mean being in red dots]{Boxplot of posterior samples of the predicted daily PM$_{10}$ ($\mu g/m^3$) and O$_{3}$ ($\mu g/m^3$) under two-pollutant missing scenario (PM$_{10}$ and O$_3$) from hourly and daily pollution models, with outliers denoted as blue crosses and the true daily mean being in red dots.}\label{fig:pm10o3missing}
\end{figure}

\begin{figure}[H]
\subfloat[Hourly model\label{fig:no2o3missing-1}]{\includegraphics[width=0.49\textwidth]{figure/no2o3missing-1} }\subfloat[Daily model\label{fig:no2o3missing-2}]{\includegraphics[width=0.49\textwidth]{figure/no2o3missing-2} }\subfloat[Hourly model\label{fig:no2o3missing-3}]{\includegraphics[width=0.49\textwidth]{figure/no2o3missing-3} }\subfloat[Daily model\label{fig:no2o3missing-4}]{\includegraphics[width=0.49\textwidth]{figure/no2o3missing-4} }\caption[Boxplot of posterior samples of the predicted daily NO$_{2}$ ($\mu g/m^3$) and O$_{3}$ ($\mu g/m^3$) under two-pollutant missing scenario (NO$_2$ and O$_3$) from hourly and daily pollution models, with outliers denoted as blue crosses and the true daily mean being in red dots]{Boxplot of posterior samples of the predicted daily NO$_{2}$ ($\mu g/m^3$) and O$_{3}$ ($\mu g/m^3$) under two-pollutant missing scenario (NO$_2$ and O$_3$) from hourly and daily pollution models, with outliers denoted as blue crosses and the true daily mean being in red dots.}\label{fig:no2o3missing}
\end{figure}

\printbibliography[title=References]

\end{document}
